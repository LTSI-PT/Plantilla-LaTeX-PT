%%Paquetes
\usepackage[T1]{fontenc} % Codificación de fuente
\usepackage{lmodern} % Fuente compatible
\usepackage{graphicx}
\usepackage{float}
%\usepackage{xcolor}
\usepackage{ragged2e}
\usepackage{amssymb}
\usepackage{algorithm}
\usepackage{algpseudocode}
\usepackage{listings}
\usepackage[table]{xcolor}

\definecolor{codegreen}{rgb}{0,0.6,0}
\definecolor{codegray}{rgb}{0.5,0.5,0.5}
\definecolor{codepurple}{rgb}{0.58,0,0.82}
\definecolor{backcolour}{rgb}{0.95,0.95,0.92}

\lstdefinestyle{mystyle}{
    backgroundcolor=\color{backcolour},   
    commentstyle=\color{codegreen},
    keywordstyle=\color{magenta},
    numberstyle=\tiny\color{codegray},
    stringstyle=\color{codepurple},
    basicstyle=\ttfamily\footnotesize,
    breakatwhitespace=false,         
    breaklines=true,                 
    captionpos=b,                    
    keepspaces=true,                 
    numbers=left,                    
    numbersep=5pt,                  
    showspaces=false,                
    showstringspaces=false,
    showtabs=false,                  
    tabsize=2
}

\lstset{style=mystyle}

% requiere texlive-science
%\usepackage{algorithm2e}
%\usepackage[ruled,vlined,spanish]{algorithm2e}
%\usepackage{alghorithmic}
%\usepackage{verbatim}
\usepackage[utf8]{inputenc}
% colores en tablas
\usepackage{colortbl}
\usepackage{array}
\usepackage[table]{xcolor}
% ligas en el índice
\usepackage{color}
\usepackage{hyperref}
\hypersetup{
    colorlinks=true,
    linkcolor=naranjauam3,
    %urlcolor=blue, %el color mas usado para ligas
    urlcolor=naranjauam2,
    linktoc=all,
    %linkbordercolor = {white}, % es el color más usado
    citecolor=naranjauam
}
%
%%% Definiciones
%
% color naranja institucional UAM-C
\definecolor{naranjauam}{HTML}{F08200}
% variaciones del naranja institucional
\definecolor{naranjauam2}{HTML}{FF4000}
\definecolor{naranjauam3}{HTML}{B43104}
\gdef \LogoUAMC{%
{\includegraphics[width=0.5\textwidth]{Plantilla/logoVar7Cua.png}\par}
}
\gdef \logosDTIDCCD{%
    {\includegraphics[width=0.25\textwidth,height=1.2cm]{Plantilla/logoCCD.png}}
    \hspace{0.15\textwidth}
    {\includegraphics[width=0.25\textwidth]{Plantilla/tsi_horizontal.png}}
    \hspace{0.15\textwidth}
    {\includegraphics[width=0.1\textwidth]{Plantilla/logoDTI.png}\par}
    
}
\gdef \asesor{\small{Asesora: Nombre completo}}
\gdef \correo#1{\texttt{#1@cua.uam.mx}}
\gdef \LTSI{%
    {\scshape\small{LIC. EN TECNOLOGÍAS Y SISTEMAS DE INFORMACIÓN \par}}%
}
\gdef \DTI{%
    {\scshape DEPARTAMENTO DE TECNOLOGÍAS DE LA INFORMACIÓN \par}
}
\gdef \dti{%
    {\scshape Departamento de Tecnologías de la Información}
}
\gdef \dccd{%
    {\scshape División de Ciencias de la Comunicación y Diseño}
}
\gdef \autor#1#2{%
    \vfill
%{\Large Autor: \par}
    {\Large #1 \par}
    {\correo{#2}}
    \vfill
    {\asesor}
    \vfill
}
\gdef \tituloPTUno#1{%
    {\scshape\Huge #1 \par}
    \vspace{2cm}
    {\itshape\Large Proyecto Terminal \par}
 }%
 \gdef \tituloPTDos#1{%
    {\scshape\Huge #1 \par}
    \vspace{
    2.5cm}
    {\itshape\Large Informe PT2 \par}
 }%
 \gdef \tituloPTTres#1{%
    {\scshape\Huge #1 \par}
    \vspace{1.5cm}
    {\itshape\Large Proyecto Terminal \par}
 }%
 \gdef \carreraDepto{%
 \DTI
\vspace{1cm}
\LTSI
 }
 
\renewcommand{\spanishabstractname}{Resumen}
\renewcommand{\spanishcontentsname}{Índice}
% Fechas en español
\renewcommand{\today}{\ifnum\number\day<10 0\fi \number\day \space%
\ifcase \month \or Enero\or Febrero\or Marzo\or Abril\or Mayo%
\or Junio\or Julio\or Agosto\or Septiembre\or Octubre\or Noviembre\or Diciembre\fi, %
\number \year}

\newcommand{\hipotesis}[1]{%
%\textbf{Hipótesis} \par
\noindent
{\color{naranjauam} \rule{\linewidth}{0.5mm}}
%\fcolorbox{white}{yellow}{%
\begin{quotation}
{\textbf{#1}\par}%
\end{quotation}
%}
\noindent
{\color{naranjauam} \rule{\linewidth}{0.5mm}}
}
% Definiciones autonumeradas
\newtheorem{definicion}{Def.}
